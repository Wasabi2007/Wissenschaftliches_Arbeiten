\chapter{A*}
Der Algorithmus von Dijkstra stellte sich als zufriedenstellend f�r den Netzwerkverkehr heraus. Aber f�r die Anforderungen, die KI-Entwickler Ende der Sechziger Jahre zu bew�ltigen hatten war er einfach nicht geeignet. Dijkstra ist klar im Vorteil wenn eine Strecke �ber bekannte Punkte mit konstanten L�ngen zur�ck gelegt werden soll. Aber sobald es m�glich ist die Kosten vom aktuelle Knoten zum Ziel zu sch�tzen, kann die Wegfindung optimiert werden.
A* ist demnach kein komplett neues Verfahren, sondern nur eine Optimierung des urspr�nglichen Algorithmus hinsichtlich dieser Punkte. Das wird besonders deutlich, wenn A* gegen eine Wand l�uft. Hier gleicht sich der Sch�tzwert, f�r neue Knoten, welche vermeidlich nahe am Ziel sind mit den ehemals h�heren Kosten der Knoten am Startpunkt an. So wird, je nach Implementierung, ein Feld nach dem anderen abgeklappert bis der Pfad nicht mehr blockiert ist.
Eine Besonderheit von A* ist, dass er optimal effizient eine L�sung findet - diese ist gleichzeitig der g�nstigste Weg. Er ist, unter den selben Ausgangsbedingungen (Heuristik), in seiner Geschwindigkeit kombiniert mit jener Genauigkeit nicht mehr zu �bertrumpfen.\footnote{ http://de.wikipedia.org/w/index.php?title=A*-Algorithmus\&oldid=129862855}\\
Dazu kommt, dass die Heuristik dahingehend modifizierbar ist, A* schneller zum Ziel finden zu lassen. Dieser Weg wird voraussichtlich nicht der optimale sein. So wird Rechenzeit auf Kosten der Genauigkeit gewonnen.

\section{Funktionsweise}
Vor jedem Schritt werden die anliegenden Knoten �berpr�ft. Der Computer berechnet zu erst, wie bei Dijkstra, die bisher zur�ck gelegten Kosten. Zus�tzlich kommt nun hinzu, dass die Punkte noch einen Sch�tzwert erhalten, wie weit sie vom Ziel entfernt sind.\\
Zur Formel kommt dies nun als der Faktor $h(k)$.\\
Es gilt nun:\\
$F(k)=D(k)+h(k)$

\section{Beispiel}

\setcounter{Schritt}{1}

\begin{figure}
\centering
\caption{Schritt 0 - 6}
\label{Schritt 0 - 6}
\begin{subfigure}[b]{0.49\textwidth} %[hbtp]
	\centering
		\includegraphics[width=\textwidth]{images/A-Stern-Beispiel/aStar(\arabic{Schritt}).jpg}
	\caption{Schritt \arabic{Schritt}}
	\label{A-Stern-Schritt \arabic{Schritt}}
	\addtocounter{Schritt}{1}
\end{subfigure}
\hfill
\begin{subfigure}[b]{0.49\textwidth} %[hbtp]
		\includegraphics[width=\textwidth]{images/A-Stern-Beispiel/aStar(\arabic{Schritt}).jpg}
	\caption{Schritt \arabic{Schritt}}
	\label{A-Stern-Schritt \arabic{Schritt}}
	\addtocounter{Schritt}{1}
\end{subfigure}
\\
\begin{subfigure}[b]{0.49\textwidth} %[hbtp]
	\centering
		\includegraphics[width=\textwidth]{images/A-Stern-Beispiel/aStar(\arabic{Schritt}).jpg}
	\caption{Schritt \arabic{Schritt}}
	\label{A-Stern-Schritt \arabic{Schritt}}
	\addtocounter{Schritt}{1}
\end{subfigure}
\hfill
\begin{subfigure}[b]{0.49\textwidth} %[hbtp]
		\includegraphics[width=\textwidth]{images/A-Stern-Beispiel/aStar(\arabic{Schritt}).jpg}
	\caption{Schritt \arabic{Schritt}}
	\label{A-Stern-Schritt \arabic{Schritt}}
	\addtocounter{Schritt}{1}
\end{subfigure}
\\
\begin{subfigure}[b]{0.49\textwidth} %[hbtp]
	\centering
		\includegraphics[width=\textwidth]{images/A-Stern-Beispiel/aStar(\arabic{Schritt}).jpg}
	\caption{Schritt \arabic{Schritt}}
	\label{A-Stern-Schritt \arabic{Schritt}}
	\addtocounter{Schritt}{1}
\end{subfigure}\\
{\small{\it{Quelle:\cite{Les:06}}}}
\end{figure}


 
