\chapter{A*}
Der Algorithmus von Dijkstra stellte sich als zufriedenstellend f�r den Netzwerkverkehr heraus. Aber f�r die Anforderungen, die KI-Entwickler Ende der sechziger Jahre, durch das Aufkommen von Computerspielen, zu bew�ltigen hatten war er einfach nicht geeignet. Dijkstra ist klar im Vorteil, wenn eine Strecke �ber bekannte Punkte mit konstanten L�ngen zur�ck gelegt werden soll. Aber sobald es m�glich ist die Kosten vom aktuelle Knoten zum Ziel zu sch�tzen, kann die Wegfindung optimiert werden.\\
A* ist demnach kein komplett neues Verfahren, sondern nur eine Optimierung des urspr�nglichen Algorithmus hinsichtlich dieser Punkte. Das wird besonders deutlich, wenn A* gegen eine Wand l�uft. Hier gleicht sich der Sch�tzwert, f�r neue Knoten, welche vermeidlich nahe am Ziel sind mit den ehemals h�heren Kosten der Knoten am Startpunkt an. So wird, je nach Implementierung, ein Feld nach dem anderen besucht, bis der Pfad nicht mehr blockiert ist.
Eine Besonderheit von A* ist, dass er optimal effizient eine L�sung findet - diese ist gleichzeitig der g�nstigste Weg.
\\ Er ist, unter den selben Ausgangsbedingungen (Heuristik), in seiner Geschwindigkeit, kombiniert mit jener Genauigkeit nicht mehr zu �bertrumpfen.\footnote{ \url{http://de.wikipedia.org/w/index.php?title=A*-Algorithmus\&oldid=129862855}}\\
Dazu kommt, dass die Heuristik dahingehend modifizierbar ist, A* schneller zum Ziel finden zu lassen. Dieser Weg wird voraussichtlich nicht der optimale sein. So wird Rechenzeit auf Kosten der Genauigkeit gewonnen.

\section{Funktionsweise}
Vor jedem Schritt werden die anliegenden Knoten �berpr�ft. Der Computer berechnet zuerst, wie bei Dijkstra, die bisher zur�ckgelegten Kosten. Zus�tzlich kommt nun hinzu, dass die Punkte noch einen Sch�tzwert erhalten, wie weit sie vom Ziel entfernt sind.\\
Zur Formel kommt dies nun als der Faktor $h(k)$.\\
Es gilt nun:\\
$F(k)=D(k)+h(k)$

\section{Beispiel}

\setcounter{Schritt}{1}

\begin{figure}
\centering
\caption{Vorgehensweise von A*}

\label{Schritt 1 - 5}

\begin{tabular}{c c}
\begin{subfigure}[b]{0.49\textwidth} %[hbtp]
	\centering
		\includegraphics[width=\textwidth]{images/A-Stern-Beispiel/aStar(\arabic{Schritt}).jpg}
	\caption{Ausgangssituation Start(Gr�n) und Ziel(Rot) werden durch das Hindernis(Blau) getrennt.}
	\label{A-Stern-Schritt \arabic{Schritt}}
	\addtocounter{Schritt}{1}
\end{subfigure}
&
\begin{subfigure}[b]{0.49\textwidth} %[hbtp]
	\centering
		\includegraphics[width=\textwidth]{images/A-Stern-Beispiel/aStar(\arabic{Schritt}).jpg}
	\caption{Untersuchen der Nachbarknoten vom Startpunkt aus. Oben links die gesch�tzten Gesamtkosten. Unten links die bisherigen Wegkosten. Unten rechts die gesch�tzte Entfernung.}
	\label{A-Stern-Schritt \arabic{Schritt}}
	\addtocounter{Schritt}{1}
\end{subfigure}
\\
\begin{subfigure}[b]{0.49\textwidth} %[hbtp]
	\centering
		\includegraphics[width=\textwidth]{images/A-Stern-Beispiel/aStar(\arabic{Schritt}).jpg}
	\caption{Die blaue Umrahmung zeigt die besuchten Knoten an. Hier endete die Suche in einer Wand. Nun wird der g�nstigste Knoten aus der Liste genommen.}
	\label{A-Stern-Schritt \arabic{Schritt}}
	\addtocounter{Schritt}{1}
\end{subfigure}
&
\begin{subfigure}[b]{0.49\textwidth} %[hbtp] 
	\centering
		\includegraphics[width=\textwidth]{images/A-Stern-Beispiel/aStar(\arabic{Schritt}).jpg}
	\caption{Der Zielknoten wurde gefunden und wird in die Liste der abgeschlossenen Knoten mit eingereiht.}
	\label{A-Stern-Schritt \arabic{Schritt}}
	\addtocounter{Schritt}{1}
\end{subfigure}
\\
\multicolumn{2}{ c  }{
\begin{subfigure}[b]{0.49\textwidth} %[hbtp]
	\centering
		\includegraphics[width=\textwidth]{images/A-Stern-Beispiel/aStar(\arabic{Schritt}).jpg}
	\caption{Nun kann der Weg vom Ziel aus durch die Verweise der einzelnen Knoten auf die Vorg�ngerknoten gebildet werden.}
	\label{A-Stern-Schritt \arabic{Schritt}}
	\addtocounter{Schritt}{1}
\end{subfigure}}
\end{tabular}
\\{\small{\it{Quelle:\cite{Les:06}}}}
\end{figure}


 
