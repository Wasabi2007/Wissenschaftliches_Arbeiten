\chapter{Review Einarbeitung}

Das Review wurde bei der finalen Abgabeversion des Dokuments ber�cksichtigt.\\
\section{Format, Grammatik, Rechtschreibung }
Die angemerkten Rechtschreib- und Satzzeichenfehler wurden korrigiert. In einigen, etwas verkorksten F�llen, haben wir die Kritik an unserer Satzstellung wahrgenommen. Dennoch sind wir der Meinung, dass auch in wissenschaftlichen Arbeiten die Hauptinformation gerne dem Nebensatz folgen darf.\\
Die Bildbeschreibungen wurden durch essentielle Aussagen erg�nzt und sind nun gleichzeitig schl�ssig und bedeutend.\\

\section{Abstrakt}
Der zweite Teil des Dokuments wird nun im Abstrakt ber�cksichtigt. Das englische Abstrakt war tats�chlich nur ein Easteregg und sollte die scharfen Augen unserer Reviewer auf die Probe stellen. Keinesfalls war die Intention irgend wen in die Irre zu f�hren.\\

\section{Inhalt, Fakten, Beispiel }
Wir hatten zu erst selbst vor auf andere Algorithmen als A* und Dijkstra einzugehen. Doch als wir in s�mtlichen Quellen nur auf Pfadfinder stie�en, welche nicht oder nur selten in Spielen angewendet werden. Haben wir uns dazu entschieden, mehr auf die Implementierungskonzepte ein zu gehen.\\
H�tten wir s�mtliche Wegfinder dahingehend begutachtet, warum diese nicht f�r Spiele geeignet sind. So h�tten wir in dem vielf�ltigen Angeboten derer, schlussendlich Kompromisse machen m�ssen. Stattdessen haben wir uns mit dem befasst was m�glich ist, anstelle dem was nicht funktioniert.\\

\section{Textanmerkungen}
Wir haben auf Seite 2 Dijkstra nun mit der Graphentheorie verkn�pft, die im Glossar erl�utert wird. Knoten sind Teil der Graphentheorie und wir gingen davon aus, dass die Leser mindestens den Wissensstand von der Veranstaltung Datenstrukturen und Algorithmen haben.\\
Zu beginn von Seite 6 wurde gefragt auf welche Genauigkeit sich bezogen wird. Im vorhergehenden Satz ist die Rede davon, dass A* immer den g�nstigsten Weg findet. Wir sehen dies als schl�ssige Verkn�pfung.\\
Bei Seite 9 wurde bem�ngelt, dass in der Abbildung nicht ersichtlich war, dass der Algorithmus erst gegen das Hindernis lief. Genau deshalb wird es dort erw�hnt. Auch wird es anhand der Abbildung \ref{lastFrameA} ersichtlich, dass A* solang an allen verf�gbaren Knoten nach einem Weg um das Hindernis gesucht hat, bis dies schlie�lich passierte.\\
Bei Seite 10: Boni und Mali betrachten wir als allgemein verwendbare W�rter.\\
Genauso auf Seite 12: Gebaren. \\
Auf Seite 14: Wir wollten ohne Codebeispiele in unserer Arbeit auskommen. Da einige Konzepte so fortgeschritten sind, dass sie sich nicht mehr mit einfachen Codebeispielen erl�utern oder erkl�ren lassen.
Wir bedanken uns bei Andr� Nonnweiler und Fabian Nola f�r die M�he, die sie sich beim Review der Arbeit gegeben haben.


