\chapter {Dijkstra}
Der auf der Graphentheorie beruhende Dijkstra bildet die Grundlage zu A*. Er wurde vom gleichnamigen Professor im Jahr 1959 in einem Lehrbuch erl�utert.\footnote{\cite{EWD:NumerMath59}} Es ist der �blichste Algorithmus, der f�r Netzwerkrouting eingesetzt wird.\footnote{\cite{tanenbaum:CN}} \\

\section {Funktionsweise}
Die Knoten im Dijkstra-Algorithmus haben folgende Eigenschaften:
\begin{itemize}
	\item Die Kosten des ersten Knotens betragen 0.
	\item Sonstige Knotenkosten akkumulieren sich aus den Kosten der Vorg�ngerknoten.
	\item Die Unbekannten ausgenommen, hat jeder Knoten ein Vorg�nger, der die bisher k�rzeste Verbindung zum ersten Knoten darstellt.
	\item fertig untersuchte Knoten werden ignoriert
\end{itemize}

\newpage
\subsection{Iterative Vorgehensweise}

\begin{enumerate}
	\item Die Knoten werden in 3 Listen sortiert
	\begin{itemize}
		\item unbekannte Knoten
		\item zu untersuchende Knoten
		\item fertig untersuchte Knoten
	\end{itemize}
	\item Am Anfang ist der Startknoten als "'zu untersuchend"' markiert, alle Anderen sind unbekannt. \ref{Schritt 0}
	\item Aus der Liste der zu untersuchenden Knoten wird nun jener erw�hlt, welcher am wenigsten Kosten verursacht. \item Dieser wird nun in die Liste der fertig untersuchten Knoten verschoben. \ref{Schritt 7}
	\begin{itemize}
		\item War dies der Zielknoten, so ergibt sich als Ergebnis der Weg vom Anfang bis zum Ziel. \ref{Schritt 25}
		\item Wenn nicht, so werden die unbekannten, direkt verbundenen Knoten in die Liste der zu Untersuchenden angef�gt. Dabei werden die Kosten der Knoten ermittelt und der Vorg�ngerknoten gesetzt.(\ref{Schritt 1} - \ref{Schritt 6})\\ 
		Es gilt:\\
		$Knotenkosten=Weg+Kante$\\
		$D(k_{i})=D(k_{i-1})+Kantenkosten$
		\\
		Falls f�r den Knoten schon einmal Kosten ermittelt wurden, so wird der niedrigste Betrag �bernommen.
	\end{itemize}
\end{enumerate}
Schritte 3 und 4 werden wiederholt, bis die Liste der zu untersuchenden Knoten leer ist. Bei einem erfolgreichen Ablauf kann vom Ziel bis zum Startknoten der k�rzeste Weg verfolgt werden.(\ref{Schritt 1} - \ref{Schritt 25})\\
Falls das Ziel nicht erreicht wurde, gibt es keinen Weg �ber die angegebenen Knoten.\\
\newpage
\subsection{Ein kleines Beispiel}

\newcounter{Schritt}
\setcounter{Schritt}{0}

\begin{figure}
\centering
\caption{Schritte vom Start bis zum Ende der Untersuchung der Nachbarknoten von 2}
\label{Schritt 0 - 11}
\begin{subfigure}[b]{0.24\textwidth} %[hbtp]
	\centering
		\includegraphics[width=\textwidth]{images/Dijkstra-Beispiel/\arabic{Schritt}.png}
	\caption{Schritt \arabic{Schritt}}
	\label{Schritt \arabic{Schritt}}
	\addtocounter{Schritt}{1}
\end{subfigure}
\hfill
\begin{subfigure}[b]{0.24\textwidth} %[hbtp]
		\includegraphics[width=\textwidth]{images/Dijkstra-Beispiel/\arabic{Schritt}.png}
	\caption{Schritt \arabic{Schritt}}
	\label{Schritt \arabic{Schritt}}
	\addtocounter{Schritt}{1}
\end{subfigure}
\hfill
\begin{subfigure}[b]{0.24\textwidth} %[hbtp]
\centering
		\includegraphics[width=\textwidth]{images/Dijkstra-Beispiel/\arabic{Schritt}.png}
	\caption{Schritt \arabic{Schritt}}
	\label{Schritt \arabic{Schritt}}
	\addtocounter{Schritt}{1}
\end{subfigure}
\hfill
\begin{subfigure}[b]{0.24\textwidth} %[hbtp]
\centering
		\includegraphics[width=\textwidth]{images/Dijkstra-Beispiel/\arabic{Schritt}.png}
	\caption{Schritt \arabic{Schritt}}
	\label{Schritt \arabic{Schritt}}
	\addtocounter{Schritt}{1}
\end{subfigure}
\\
\begin{subfigure}[b]{0.24\textwidth} %[hbtp]
	\centering
		\includegraphics[width=\textwidth]{images/Dijkstra-Beispiel/\arabic{Schritt}.png}
	\caption{Schritt \arabic{Schritt}}
	\label{Schritt \arabic{Schritt}}
	\addtocounter{Schritt}{1}
\end{subfigure}
\hfill
\begin{subfigure}[b]{0.24\textwidth} %[hbtp]
		\includegraphics[width=\textwidth]{images/Dijkstra-Beispiel/\arabic{Schritt}.png}
	\caption{Schritt \arabic{Schritt}}
	\label{Schritt \arabic{Schritt}}
	\addtocounter{Schritt}{1}
\end{subfigure}
\hfill
\begin{subfigure}[b]{0.24\textwidth} %[hbtp]
\centering
		\includegraphics[width=\textwidth]{images/Dijkstra-Beispiel/\arabic{Schritt}.png}
	\caption{Schritt \arabic{Schritt}}
	\label{Schritt \arabic{Schritt}}
	\addtocounter{Schritt}{1}
\end{subfigure}
\hfill
\begin{subfigure}[b]{0.24\textwidth} %[hbtp]
\centering
		\includegraphics[width=\textwidth]{images/Dijkstra-Beispiel/\arabic{Schritt}.png}
	\caption{Schritt \arabic{Schritt}}
	\label{Schritt \arabic{Schritt}}
	\addtocounter{Schritt}{1}
\end{subfigure}
\\
\begin{subfigure}[b]{0.24\textwidth} %[hbtp]
	\centering
		\includegraphics[width=\textwidth]{images/Dijkstra-Beispiel/\arabic{Schritt}.png}
	\caption{Schritt \arabic{Schritt}}
	\label{Schritt \arabic{Schritt}}
	\addtocounter{Schritt}{1}
\end{subfigure}
\hfill
\begin{subfigure}[b]{0.24\textwidth} %[hbtp]
		\includegraphics[width=\textwidth]{images/Dijkstra-Beispiel/\arabic{Schritt}.png}
	\caption{Schritt \arabic{Schritt}}
	\label{Schritt \arabic{Schritt}}
	\addtocounter{Schritt}{1}
\end{subfigure}
\hfill
\begin{subfigure}[b]{0.24\textwidth} %[hbtp]
\centering
		\includegraphics[width=\textwidth]{images/Dijkstra-Beispiel/\arabic{Schritt}.png}
	\caption{Schritt \arabic{Schritt}}
	\label{Schritt \arabic{Schritt}}
	\addtocounter{Schritt}{1}
\end{subfigure}
\hfill
\begin{subfigure}[b]{0.24\textwidth} %[hbtp]
\centering
		\includegraphics[width=\textwidth]{images/Dijkstra-Beispiel/\arabic{Schritt}.png}
	\caption{Schritt \arabic{Schritt}}
	\label{Schritt \arabic{Schritt}}
	\addtocounter{Schritt}{1}
\end{subfigure}
\end{figure}
\begin{figure}
\caption{2 wird in die "'abgeschlossen Liste"' integriert, bis "'Pfad gefunden"'}
\label{Schritt 12 - 26}
\begin{subfigure}[b]{0.24\textwidth} %[hbtp]
	\centering
		\includegraphics[width=\textwidth]{images/Dijkstra-Beispiel/\arabic{Schritt}.png}
	\caption{Schritt \arabic{Schritt}}
	\label{Schritt \arabic{Schritt}}
	\addtocounter{Schritt}{1}
\end{subfigure}
\hfill
\begin{subfigure}[b]{0.24\textwidth} %[hbtp]
		\includegraphics[width=\textwidth]{images/Dijkstra-Beispiel/\arabic{Schritt}.png}
	\caption{Schritt \arabic{Schritt}}
	\label{Schritt \arabic{Schritt}}
	\addtocounter{Schritt}{1}
\end{subfigure}
\hfill
\begin{subfigure}[b]{0.24\textwidth} %[hbtp]
\centering
		\includegraphics[width=\textwidth]{images/Dijkstra-Beispiel/\arabic{Schritt}.png}
	\caption{Schritt \arabic{Schritt}}
	\label{Schritt \arabic{Schritt}}
	\addtocounter{Schritt}{1}
\end{subfigure}
\hfill
\begin{subfigure}[b]{0.24\textwidth} %[hbtp]
\centering
		\includegraphics[width=\textwidth]{images/Dijkstra-Beispiel/\arabic{Schritt}.png}
	\caption{Schritt \arabic{Schritt}}
	\label{Schritt \arabic{Schritt}}
	\addtocounter{Schritt}{1}
\end{subfigure}
\\
\begin{subfigure}[b]{0.24\textwidth} %[hbtp]
	\centering
		\includegraphics[width=\textwidth]{images/Dijkstra-Beispiel/\arabic{Schritt}.png}
	\caption{Schritt \arabic{Schritt}}
	\label{Schritt \arabic{Schritt}}
	\addtocounter{Schritt}{1}
\end{subfigure}
\hfill
\begin{subfigure}[b]{0.24\textwidth} %[hbtp]
		\includegraphics[width=\textwidth]{images/Dijkstra-Beispiel/\arabic{Schritt}.png}
	\caption{Schritt \arabic{Schritt}}
	\label{Schritt \arabic{Schritt}}
	\addtocounter{Schritt}{1}
\end{subfigure}
\hfill
\begin{subfigure}[b]{0.24\textwidth} %[hbtp]
\centering
		\includegraphics[width=\textwidth]{images/Dijkstra-Beispiel/\arabic{Schritt}.png}
	\caption{Schritt \arabic{Schritt}}
	\label{Schritt \arabic{Schritt}}
	\addtocounter{Schritt}{1}
\end{subfigure}
\hfill
\begin{subfigure}[b]{0.24\textwidth} %[hbtp]
\centering
		\includegraphics[width=\textwidth]{images/Dijkstra-Beispiel/\arabic{Schritt}.png}
	\caption{Schritt \arabic{Schritt}}
	\label{Schritt \arabic{Schritt}}
	\addtocounter{Schritt}{1}
\end{subfigure}
\\
\begin{subfigure}[b]{0.24\textwidth} %[hbtp]
	\centering
		\includegraphics[width=\textwidth]{images/Dijkstra-Beispiel/\arabic{Schritt}.png}
	\caption{Schritt \arabic{Schritt}}
	\label{Schritt \arabic{Schritt}}
	\addtocounter{Schritt}{1}
\end{subfigure}
\hfill
\begin{subfigure}[b]{0.24\textwidth} %[hbtp]
		\includegraphics[width=\textwidth]{images/Dijkstra-Beispiel/\arabic{Schritt}.png}
	\caption{Schritt \arabic{Schritt}}
	\label{Schritt \arabic{Schritt}}
	\addtocounter{Schritt}{1}
\end{subfigure}
\hfill
\begin{subfigure}[b]{0.24\textwidth} %[hbtp]
\centering
		\includegraphics[width=\textwidth]{images/Dijkstra-Beispiel/\arabic{Schritt}.png}
	\caption{Schritt \arabic{Schritt}}
	\label{Schritt \arabic{Schritt}}
	\addtocounter{Schritt}{1}
\end{subfigure}
\hfill
\begin{subfigure}[b]{0.24\textwidth} %[hbtp]
\centering
		\includegraphics[width=\textwidth]{images/Dijkstra-Beispiel/\arabic{Schritt}.png}
	\caption{Schritt \arabic{Schritt}}
	\label{Schritt \arabic{Schritt}}
	\addtocounter{Schritt}{1}
\end{subfigure}
\\
\begin{subfigure}[b]{0.24\textwidth} %[hbtp]
	\centering
		\includegraphics[width=\textwidth]{images/Dijkstra-Beispiel/\arabic{Schritt}.png}
	\caption{Schritt \arabic{Schritt}}
	\label{Schritt \arabic{Schritt}}
	\addtocounter{Schritt}{1}
\end{subfigure}
\hfill
\begin{subfigure}[b]{0.24\textwidth} %[hbtp]
		\includegraphics[width=\textwidth]{images/Dijkstra-Beispiel/\arabic{Schritt}.png}
	\caption{Schritt \arabic{Schritt}}
	\label{Schritt \arabic{Schritt}}
	\addtocounter{Schritt}{1}
\end{subfigure}
\hfill
\begin{subfigure}[b]{0.24\textwidth} %[hbtp]
\centering
		\includegraphics[width=\textwidth]{images/Dijkstra-Beispiel/\arabic{Schritt}.png}
	\caption{Schritt \arabic{Schritt}}
	\label{Schritt \arabic{Schritt}}
	\addtocounter{Schritt}{1}
\end{subfigure}
\hfill
\end{figure}
{\small{\it{Quelle: http://upload.wikimedia.org/wikipedia/commons/5/57/Dijkstra\_Animation.gif}}}