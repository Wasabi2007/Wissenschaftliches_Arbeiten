%%%%%%%%%%%%%%%%%%% vorlage.tex %%%%%%%%%%%%%%%%%%%%%%%%%%%%%
%
% Beispiel-Vorlage zur Erstellung von Projekt-Dokumentationen
%
% Benutzen Sie bitte diese Datei um Ihre Dokumente zu erstellen
%
%%%%%% erstellt anhand svmono-Springer-Verlag-Vorlage %%%%%%%%%
%Hier hat Emolotow etwas Senf hinterlassen =D

%%%%%%%%%%%%%%%%%%%%%%%%%%%%%%%%%%%%%%%%%%%%%%%%%%%
\documentclass[envcountsame,envcountchap, deutsch]{i-studis}


\usepackage{makeidx}         % Erlaubt die Erzeugung eines Index-Verzeichnisses
\usepackage{multicol}        % Zweispaltiger Index-Verzeichnis
%\usepackage[bottom]{footmisc} % Erzeugung von Fu�noten nur beim Bedarf einbinden
\usepackage{caption}
\usepackage{subcaption}

%%-----------------------------------------------------
%\newif\ifpdf
%\ifx\pdfoutput\undefined
%\pdffalse
%\else
%\pdfoutput=1
%\pdftrue
%\fi
%%--------------------------------------------------------
%\ifpdf
\usepackage[pdftex]{graphicx}
\usepackage[pdftex,plainpages=false]{hyperref}
%\else
%\usepackage{graphicx}
%\usepackage[plainpages=false]{hyperref}
%\fi
%%-----------------------------------------------------
\usepackage{color}	% Farbverwaltung
%\usepackage{ngerman} % Neue deutsche Rechtsschreibung
\usepackage[english, ngerman]{babel}
\usepackage[latin1]{inputenc} % Erm�glicht Umlaute-Darstellung
%\usepackage[utf8]{inputenc}  % Ermöglicht Umlaute-Darstellung unter Linux (je nach verwendetem Format)
%-----------------------------------------------------
\usepackage{listings} % Code-Darstellung
\usepackage{media9} %Gif and Movie support
\lstset
{% general command to set parameter(s)
	basicstyle=\scriptsize, % print whole listing small
	keywordstyle=\color{blue}\bfseries,
	% underlined bold black keywords
	identifierstyle=, % nothing happens
	commentstyle=\color{red}, % white comments
	stringstyle=\ttfamily, % typewriter type for strings
	showstringspaces=false, % no special string spaces
	framexleftmargin=7mm, 
	tabsize=3,
	showtabs=false,
	frame=single, 
	rulesepcolor=\color{blue},
	numbers=left,
	linewidth=146mm,
	xleftmargin=8mm
}
\usepackage{textcomp} % celsius - Darstellung
\usepackage{
amssymb,
amsfonts,
amstext,
amsmath
} % Mathematische Symbole
\usepackage[german, ruled, vlined]{algorithm2e}
\usepackage[a4paper]{geometry} %Andere Formatierung
\usepackage{bibgerm}
\usepackage{array}
\hyphenation{Ele-men-tar-ob-jek-te  ab-ge-tas-tet Aus-wer-tung House-holder-Matrix Le-ast-Squa-res-Al-go-ri-th-men} %Silbentrennung bei falschen Trennung angeben
\setlength{\textheight}{1.1\textheight}
\pagestyle{myheadings} % Erzeugt selbstdefinierte Kopfzeile
\makeindex % Index-Erstellung

% Ab hier beginnt das eigentliche Dokument.
%--------------------------------------------------------------------------
\begin{document}

%------------------------- Titelblatt -------------------------------------
\title{\textbf{Wohin geht?}}
\subtitle{Where go?}
%---- Die Art der Dokumentation kann hier ausgew�hlt werden---------------
%\project{Master Abschlussarbeit}
%\project{Master Projektstudium}
%\project{Bachelor Abschlussarbeit}
%\project{Projektarbeit}
%\project{Seminar zur Vorlesung ...}
\project{Wissenschaftliches Arbeiten}
%--------------------------------------------------------------------------
\supervisor{Titel Vorname Name} % Betreuer der Arbeit
\author{Boos, Jeremias \\ Neugebauer, Manuel} %Autor der Arbeit
\address{\today ,Trier} % Im Zusammenhang mit dem Datum wird hinter dem Ort ein Komma angegeben
\submitdate{Abgabedatum} % Abgabedatum
%\begingroup
%  \renewcommand{\thepage}{title}
%  \mytitlepage
%  \newpage
%\endgroup
\begingroup
  \renewcommand{\thepage}{Titel}
  \mytitlepage
  \newpage
\endgroup
%--------------------------------------------------------------------------
\frontmatter 
%--------------------------------------------------------------------------
\danksagung

\centering
Vielen Dank an das TexStudio Team\\
besonders an den Entwickler, welcher die Funktion 
\\"'zuf�lligen Text generieren"'\\
erfunden hat.
\flushleft %Danksagungen
?Danke f�r diesen guten Morgen!?
\input{chapters/Vorwort}
%%%%%%%%%%%%%%%%%%%%%% Kurzfassung.tex %%%%%%%%%%%%%%%%%%%%%%%%%%%%%%%%%%%%%
%
% sample preface
%
% Use this file as a template for your own input.
%
%%%%%%%%%%%%%%%%%%%%%%%% Spinner-Verlag %%%%%%%%%%%%%%%%%%%%%%%%%%

\kurzfassung

%% Schreiben Sie Ihre Kurzfassung hierher
A* ist der meistbenutzte Algorithmus in Sachen Wegfindung. Es werden die Funktionen und Berechnungen dahinter erkl�rt. Auch wird auf die Grundlage dessen eingegangen.
\\
Danach folgt ein Einblick in Konzepte zur effizienten und nat�rlichen Umsetzung von Wegfindealgorithmen in Spielen. 
\\[3ex]

%% Please write your preface here
%this is just to troll the reviewer
%\noindent
%A* is the most commonly used algorythm for pathfinding. Here we are, to praise the holy god of A*, that he will crush our enemys and bless our code. Hail to the glorious algorythm who gives us CPU time. \\obey!!

 % Kurzfassung Deutsch/English
\tableofcontents %Inhaltsverzeichnis - notwendig
\listoffigures % Abbildungsverzeichnis - optional
\listoftables % Tabellenverzeichnis- optional
%--------------------------------------------------------------------------
\mainmatter                        %Hauptteil (ab hier arab. Seitenzahlen)
%--------------------------------------------------------------------------
% Kapitell werden einzeln abgespeichert und hier eingef�gt
\chapter{Einleitung}
%Begonnen werden soll mit einer Einleitung zum Thema: z.B. Hintergrund und Ziel
%(was, warum).
In vielen Computerspielen ist eine Funktion von N�ten um Einheiten und Charaktere ans Ziel zu f�hren. Egal in welchen Genre, wenn sich der Computer in einer simulierten Umgebung effizient zurecht finden soll, wird ein Wegfindealgorithmus ben�tigt. Um Umwege und unnat�rlich wirkendes Verhalten zu vermeiden, bietet sich ein heuristischer, Graphen basierter Algorithmus an.
Dieser hat sich als effizienteste L�sung f�r Wegfindeprobleme erwiesen.\\
Nicht nur in Civilization, wo die Funktionsweise dem Spieler als ein Zentrales Spielelement bewusst wird, finden solche Algorithmen Verwendung. Auch Genres scheinbar abseits taktischer Tiefe, wie Beispielsweise MMORPGS oder Shootern w�ren ohne dieselben nicht realisierbar.\\

%In vielen Jahren der Spieleentwicklung hat sich A*(A-Stern) als der Effizienteste seiner Art herauskristallisiert.
Wegen seiner einfachen Implementierung und seines hohen Effizienzgrades hat sich A*(A-Stern) als Geeignetster seiner Art bew�hrt. Daher wird heute kaum noch auf andere L�sungen zur�ckgegriffen um entsprechende Probleme zu bew�ltigen. In so fern wird sich diese Arbeit mit den Eigenschaften und der Umsetzung in Programmen von A* befassen.


%\input{chapters/Problemstellung}
%\input{chapters/Aufgabenstellung}
%\input{chapters/Inhalt}
%\input{chapters/Stile}
%\input{chapters/Beispiel}
% ...

%--------------------------------------------------------------------------
\backmatter                        %Anhang
%-------------------------------------------------------------------------
%Literaturverzeichnis - notwendig
%Literaturverzeichnis - notwendig
% - use bibstyle 'geralpha', 'gerplain', ...
%\bibliographystyle{gerplain}
\bibliographystyle{geralpha}
\bibliography{literatur}     %BibTeX-File literatur.bib
%--------------------------------------------------------------------------
\printindex % Index ausdr�cken - optional
%--------------------------------------------------------------------------
% Anh�nge sind optional
\begin{appendix}
   \chapter{Glossar}

%geordnete tabelle, 2spaltig?
%\listofabbreviations
\abbreviation{Heuristik}		{Eine Methode, die ein Programm durch Logik optimiert.}
\abbreviation{Civilisation}		{Rundenbasiertes Aufbauspiel mit einem Raster als Spielfl�che.	}
\abbreviation{MMORPG} 			{Onlinerollenspiel mit sehr vielen Spielern auf einem Server(Massive Multiplayer Roll-Playing-Game)}
\abbreviation{Shooter}		{Ein Spiel mit Feuerwaffen, welches aus der Ego-Perspektive gespielt wird.}
\abbreviation{Mob}		{Alle sich bewegenden Spielobjekte(Moving Object)}
\abbreviation{NPC}		{Nicht Spielercharakter (Non-Player-Charakter)}
\abbreviation{cheaten}		{Schummeln, speziell hier in Videospielen}
\abbreviation{Graphentheorie} {Eine M�glichkeit Probleme auf Knoten und Kanten herunter zu brechen}   
   \include{chapters/AnhangB} 
\end{appendix}
\end{document}
